\documentclass{article}
    % General document formatting
    \usepackage[margin=0.7in]{geometry}
    \usepackage[parfill]{parskip}
    \usepackage[utf8]{inputenc}
    % Related to math
    \usepackage{amsmath,amssymb,amsfonts,amsthm}
    \renewcommand{\baselinestretch}{1.5}
\title{Reportes Semanales}
\author{Andrea Sánchez}
\begin{document}
\maketitle
    \textbf{Semana del 27 al 31 de Enero 2020:}\newline 
    La meta principal para la primera semana de investigación fue preparar las herramientas computacionales para poder realizar las actividades planeadas a futuro, tales como la utilización de análisis estadístico para obtener información de datos empleando diferentes modelos matemáticos.

    El propósito de esto fue presentar, estudiar y detallar cada paso del seguimiento que se escolta para realizar una tarea. Las herramientas computacionales distinguidas que ofrecen orden y estructura al equipo son Git\footnote{Sistema de control de versiones.~\cite{bergh2019}} y Docker\footnote{Proyecto de código abierto que automatiza el despliegue de aplicaciones dentro de contenedores de software.~\cite{bergh2019}}.
        Estas heramientas son utilizadas para el análisis de datos eficiente ya que segmenta el proceso en pequeñas componentes.

    Al utilizar Docker generamos un contenedor donde es posible trabajar bajo el mismo entorno, incluso cuando cada miembro labora con sistemas operativos diferentes. Por otro lado, en Git obtenemos las instrucciones que debe seguir el contenedor y así represente al equipo una plataforma ideal para terminar la tarea. Por ultimo, con GitKraken\footnote{Interfaz gráfica multiplataforma para Git desarrollada con Electron.~\cite{bergh2019}}, llevamos un registro de los cambios en los archivos dentro de un repositorio compartiendolos con todo el equipo, de modo que, se trabaja en conjunto en una sola tarea.
    El objetivo de todo lo anterior es poder trabajar de manera interactiva y remota con el equipo de Ciencia de Datos, reduciciendo el tiempo en el ciclo de análisis para mejorar la calidad.


\begin{thebibliography}{1} 
\bibitem{bergh2019} Bergh, C., Benghiat, G., \& Strod, E. (2019). The Data OPS Cookbook.. Cambridge, MA, EUA: DataKitchen Headquarters.
\end{thebibliography}
\end{document}